\documentclass[12pt,a4paper]{amsart}
\usepackage[slovene]{babel}
\usepackage[utf8]{inputenc}
\usepackage[T1]{fontenc}
\usepackage{amsmath,amssymb,amsfonts}
\usepackage{url}
\usepackage[dvipsnames,usenames]{color}
\usepackage{caption}
\usepackage{lipsum}
\usepackage{tikz}
\usepackage{xcolor}

\usetikzlibrary{graphs}
\usetikzlibrary{graphs.standard}


% oblika strani
\textwidth 15cm
\textheight 24cm
\oddsidemargin.5cm
\evensidemargin.5cm
\topmargin-5mm
\addtolength{\footskip}{10pt}
\pagestyle{plain}
\overfullrule=15pt % oznaci predlogo vrstico


% ukazi za matematicna okolja
\theoremstyle{definition} 
\newtheorem{definicija}{Definition}[section]
\newtheorem{primer}[definicija]{Example}
\newtheorem{opomba}[definicija]{Remark}

\renewcommand\endprimer{\hfill$\diamondsuit$}

\theoremstyle{plain}
\newtheorem{lema}[definicija]{Lemma}
\newtheorem{izrek}[definicija]{Theorem}
\newtheorem{trditev}[definicija]{Statement}
\newtheorem{posledica}[definicija]{Corollary}
\newtheorem{conjecture}[definicija]{Conjecture}

% ukaz za slovarsko geslo
\newlength{\odstavek}
\setlength{\odstavek}{\parindent}

% novi ukazi 
\newcommand{\program}{Financial mathematics}
\newcommand{\imeavtorja}{Jon Pascal Miklavčič, Nik Živkovič Kokalj}
\newcommand{\imementorja}{Assist.~Prof.~Dr.~Janoš Vidali}
\newcommand{\imesomentorja}{Prof.~Dr.~Riste Škrekovski}
\newcommand{\naslovdela}{Packing Coloring of Subcubic Planar Graphs}
\newcommand{\letnica}{2024}

\begin{document}

\thispagestyle{empty}
{\large
\noindent UNIVERSITY OF LJUBLJANA\\[1mm]
FACULTY OF MATHEMATICS AND PHYSICS\\[5mm]
\program\ }
\vfill

\begin{center}{\large
\imeavtorja\\[2mm]
{\bf \Large \naslovdela}\\[10mm]
{\normalsize Term Paper in Finance Lab}\\[1mm]
{\normalsize Long Presentation}\\[1cm]
{\normalsize Advisers:}\\
{\normalsize \imementorja, \\ \imesomentorja}\\[2mm]}
\end{center}
\vfill

{\large Ljubljana, \letnica}
\pagebreak

\section*{Generating graphs}

\subsection*{Function: \texttt{removable\_vertices}}

The function \texttt{removable\_vertices(G)} identifies vertices in a given graph \( G \) that can be removed while 
maintaining the connectivity of the graph.

We firstly initialize an empty list to store removable vertices. The function then iterates through each vertex \( v \) 
in the vertex set of \( G \). Fore each iteration a copy \( H \) of \( G \) is created to prevent modifying the original
graph. After that vertex \( v \) is removed from a copy. If \( H \) remains connected after the removal of \( v \), then 
\( v \) is appended to the \texttt{removable} list. At the end the function returns the list of removable vertices.


\subsection*{Function: \texttt{modify\_planar\_subcubic\_graph}} 

The function \\
\texttt{modify\_planar\_subcubic\_graph(G)} modifies a planar subcubic graph \( G \) into a new subcubic 
planar graph while ensuring that the total number of vertices remains constant.

\begin{enumerate}
    \item A copy of \( G \) is created to avoid altering the original input.
    \item The function verifies that \( G \) is both planar and subcubic. If not, function raises an error.
    \item The set of faces in \( G \) is retrieved. If no faces exist, the function returns \( G \) unchanged.
    \item A random face is selected, and its vertices are extracted into a list \\
    'face\_vertices'.
    \item A random number (between 1 and 3) of edges in the face is selected for subdivision. Subdivision is the insertion 
    of a new vertex in the middle of an exiting edge, which keeps graph subcubic and planar.
    \begin{itemize}
        \item A list of edges within the face is created. If no such edges exist, the subdivision step is skipped.
        \item A random edge \( (a, b) \) is chosen.
        \item A new vertex is introduced between \( a \) and \( b \) and connected to each one of them, while edge 
        \( (a, b) \) 
        \item The function \texttt{removable\_vertices} is called to check if any vertex can be removed.
        \item If a removable vertex exists, a random one is deleted.
        \item If no removable vertex is found, the newly inserted vertex is deleted, and the original edge \( (a, b) \) 
        is restored since we want to maintain the number of vertices in the input graph the same as in the modified graph.
    \end{itemize}
    \item A new vertex is introduced:
    \begin{itemize}
        \item The set of eligible vertices (those of degree at most 2) within the selected face is identified. These 
        vertices are eligible for a new vertex to be connected to them. We only choose vertices from a selected face
        because we do not want to compromise planarity.
        \item A new vertex is added to \( G \).
        \item The new vertex is connected to a subset of eligible vertices with one, two or three edges.
        \item The function \texttt{removable\_vertices} is used again to check for possible vertex removals.
        \item If a removable vertex exists, a random one is deleted.
        \item If no removable vertex is found, the newly added vertex is deleted.
    \end{itemize}
    \item The modified graph \( G \) is returned.
\end{enumerate}

This approach ensures that the graph remains planar and subcubic while making modifications that preserve its connectivity.


\end{document}