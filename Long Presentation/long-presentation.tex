\documentclass[12pt,a4paper]{amsart}
\usepackage[slovene]{babel}
\usepackage[utf8]{inputenc}
\usepackage[T1]{fontenc}
\usepackage{amsmath,amssymb,amsfonts}
\usepackage{url}
\usepackage[dvipsnames,usenames]{color}
\usepackage{caption}
\usepackage{lipsum}
\usepackage{tikz}
\usepackage{xcolor}

\usetikzlibrary{graphs}
\usetikzlibrary{graphs.standard}


% oblika strani
\textwidth 15cm
\textheight 24cm
\oddsidemargin.5cm
\evensidemargin.5cm
\topmargin-5mm
\addtolength{\footskip}{10pt}
\pagestyle{plain}
\overfullrule=15pt % oznaci predlogo vrstico


% ukazi za matematicna okolja
\theoremstyle{definition} 
\newtheorem{definicija}{Definition}[section]
\newtheorem{primer}[definicija]{Example}
\newtheorem{opomba}[definicija]{Remark}

\renewcommand\endprimer{\hfill$\diamondsuit$}

\theoremstyle{plain}
\newtheorem{lema}[definicija]{Lemma}
\newtheorem{izrek}[definicija]{Theorem}
\newtheorem{trditev}[definicija]{Statement}
\newtheorem{posledica}[definicija]{Corollary}
\newtheorem{conjecture}[definicija]{Conjecture}

% ukaz za slovarsko geslo
\newlength{\odstavek}
\setlength{\odstavek}{\parindent}

% novi ukazi 
\newcommand{\program}{Financial mathematics}
\newcommand{\imeavtorja}{Jon Pascal Miklavčič, Nik Živkovič Kokalj}
\newcommand{\imementorja}{Assist.~Prof.~Dr.~Janoš Vidali}
\newcommand{\imesomentorja}{Prof.~Dr.~Riste Škrekovski}
\newcommand{\naslovdela}{Packing Coloring of Subcubic Planar Graphs}
\newcommand{\letnica}{2024}

\begin{document}

\thispagestyle{empty}
{\large
\noindent UNIVERSITY OF LJUBLJANA\\[1mm]
FACULTY OF MATHEMATICS AND PHYSICS\\[5mm]
\program\ }
\vfill

\begin{center}{\large
\imeavtorja\\[2mm]
{\bf \Large \naslovdela}\\[10mm]
{\normalsize Term Paper in Finance Lab}\\[1mm]
{\normalsize Long Presentation}\\[1cm]
{\normalsize Advisers:}\\
{\normalsize \imementorja, \\ \imesomentorja}\\[2mm]}
\end{center}
\vfill

{\large Ljubljana, \letnica}
\pagebreak

To obtain the best possible packing chromatic number, we needed to generate as many different subcubic planar graphs as 
possible.  To achieve this, we made a function named modify\_planar\_subcubic\_graph(G) which takes subcubic planar graph $G$ 
as an input and returnsdifferent subcubic planar graph.

Description of a function modify\_planar\_subcubic\_graph(G):\\

Firstly, we ensure that the entered graph is subcubic and planar. If not, the function returns an error. We also added a 
condition that the graph must be connected. This is because, for the packing coloring number, if the graph is composed of 
multiple disconnected parts, the packing coloring number of the graph will equal the packing coloring number of the part 
that has the biggest packing coloring number. This ensures that the function is not as time-consuming as it would otherwise 
be. Now, we are certain that we are operating on a subcubic and planar graph. We then defined three different operations 
on the graph. In each iteration, one of them is executed. Which operation will be chosen is determined randomly using the 
function "choice" from the random library. \\

We named the first operation add\_vertex. A new vertex is generated and added to the graph. 
We refer to the added vertex as $e$. Again, using the function "choice", the algorithm selects a random edge in the 
given graph, for instance, the edge $(u, v)$. It then generates two new edges, $(e, u)$ and $(e, v)$, and deletes the 
existing edge $(u, v)$. Since we are essentially adding a vertex on to an existing edge, planarity is preserved. The graph 
remains subcubic because the vertex degrees remain unchanged. The graph also remains connected. \\

The second operation is rewire\_edge. The algorithm selects a random edge in the graph $G$ and checks if the graph, 
without the chosen edge, remains connected. If so, the edge is removed. This process is repeated until two edges are removed 
or until there are no appropriate edges left to remove. This means that if no edges can be removed, or only one can be removed, 
then none or only one edge will be removed. After this step, two new edges are randomly added. This is done by sampling 
two random vertices and generating a new edge between them. If the resulting graph meets the requirements of being planar 
and subcubic, the new edge is added to the graph. Since adding edges cannot destroy the connectivity of a graph, we do not 
need to check for connectivity. \\

The third and final operation is face. The algorithm chooses a random face of the entered graph $G$. If there 
are no faces to choose from, the function returns an error. Selecting a face ensures that the graph remains planar, subcubic, 
and connected. After the face is chosen, we label this face as a new graph $f$. We then add a new vertex to $f$ and refer 
to this vertex as $v$. We want to connect vertex $v$ to the graph $f$. This is done by adding edges from $v$ to vertices 
in $f$. Firstly, we check how many eligible vertices exist in $f$ that can be connected to $v$.\\

We consider the following cases:

\begin{enumerate}
    \item There are less than two eligible vertices in $f$ that can be connected to $v$: \\
    If there is one or zero eligible vertices, we proceed as in the operation \texttt{add\_vertex} by placing vertex $v$ 
    on an existing edge. Since only one edge is removed and two are added, we search for another edge to remove while 
    maintaining the connectivity of the graph. If no such edge exists, only one edge will be removed, and two will be added.

    \item There are two or more eligible vertices in $f$ that can be connected to $v$:\\
    In this case, using the function \texttt{randint}, we randomly select two or three eligible vertices to connect to $v$. 
    If there are only two eligible vertices, there is no need for random selection. After selecting the vertices, we connect 
    them to $v$. Since we operate on a face of the graph, planarity is preserved. Because we added a vertex and edges, we 
    also aim to remove as many edges as we added. This is achieved by choosing random edges to remove while maintaining connectivity.
\end{enumerate}

Simply choosing a face reduces the graph's complexity, serving as a kind of reset.

\end{document}